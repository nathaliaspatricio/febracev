%consideracoes_finais.tex

\chapter{Considerações Finais}

\section{Trabalhos futuros}

São vislumbradas alguns trabalhos futuros que podem ser desenvolvidos em cima desse projeto. 

Uma dessas possíveis continuações que pode ser constatada através dos dados da pesquisa de perfil de usuário é a da existência de funcionalidades que dêem suporte para que um projeto possa ser feito a distância, agregando participantes de diferentes localidades. Sendo assim, faz-se necessária uma maior integração com o sistema de submissão de projetos da FEBRACE, visto que projetos desenvolvidos com apoio da rede social podem ser submetidos pesteriormente para a participação na feira física. 

Atualmente, a rede social só apresenta a possibilidade de exposição de projetos que já foram apresentados em edições anteriores da feira. Uma possibilidade é haver uma pré seleção para a exposição virtual de projetos e dentre desses sejam escolhidos os finalistas que irão expôr seus projetos na FEBRACE. Assim, seria aberta a possibilidade de mais pessoas poderem mostrar seus projetos, mesmo que seja apenas no mundo virtual, já que são submetidos muito mais projetos do que aqueles que podem ser aprovados devido a restrição de espaço físico para comportá-los. Porém, muitos desses reprovados apresentam boa qualidade, ficando com notas da pré-avaliação muito próximas dos que foram aprovados. Além disso, alguns projetos aprovados não vêem para a feira por falta de recursos financeiros para arcar com os custos do deslocamento físico.

Como esse projeto será dividido em duas etapas, na segunda serão feitos estudos sobre a usabilidade da rede social desenvolvida e levantadas melhorias a serem implementadas em versões posteriores para que esta se adeque melhor ao uso das pessoas. Também será estudada a questão da acessibilidade do sistema, não só para a inclusão de usuários com deficiência visual, mas também para usuários usando diferentes dispositivos para o acesso a rede social. Tendo que em vista a baixa penetração do uso de internet em dispositivos móveis, como pode ser constatado no resultados das pesquisas realizadas, projeta-se um crescimento desse uso para os próximos anos.


