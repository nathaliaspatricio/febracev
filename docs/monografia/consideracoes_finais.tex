%consideracoes_finais.tex

\chapter{Considerações Finais}

O projeto FEBRACE\textsuperscript{V} têm um grande potencial de crescimento e abrangência, não só porque há cada ano estão envolvidas direta e indiretamente cerca de 13 mil pessoas com a FEBRACE, mas também porque parte da premissa que “tecnologias de sucesso são aquelas que se harmonizam com as necessidades e sustentam os relacionamentos e as atividades que enriquecem as experiências dos usuários” \cite{shneiderman06}.

Como foi dito anteriormente, a FEBRACE possui uma grande importância no incentivo à investigação científica e desenvolvimento tecnológico por jovens e o projeto aqui tratado extende ao mundo virtual essa atuação. Por isso, em geral, os impactos sociais decorrentes da utilização da rede social são positivos, promovendo um crescimento pessoal de seus participantes.

Na questão técnica, a metodologia de desenvolvimento se mostrou bastante aderente à necessidade do projeto e, com base em experiências em desenvolvimentos anteriores, apresentou vantagens em relação às metodologias mais tradicionais. A escolha da metodologia usada foi decisiva para o sucesso do projeto, tendo em visto o curto tempo para seu desenvolvimento.

\section{Trabalhos futuros}

São vislumbradas alguns trabalhos futuros que podem ser desenvolvidos em cima desse projeto. 

Uma dessas possíveis continuações que pode ser constatada através dos dados da pesquisa de perfil de usuário é a da existência de funcionalidades que dêem suporte para que um projeto possa ser feito a distância, agregando participantes de diferentes localidades. Sendo assim, faz-se necessária uma maior integração com o sistema de submissão de projetos da FEBRACE, visto que projetos desenvolvidos com apoio da rede social podem ser submetidos pesteriormente para a participação na feira física. 

Atualmente, a rede social só apresenta a possibilidade de exposição de projetos que já foram apresentados em edições anteriores da feira. Uma possibilidade é haver uma pré seleção para a exposição virtual de projetos e dentre desses sejam escolhidos os finalistas que irão expôr seus projetos na FEBRACE. Assim, seria aberta a possibilidade de mais pessoas poderem mostrar seus projetos, mesmo que seja apenas no mundo virtual, já que são submetidos muito mais projetos do que aqueles que podem ser aprovados devido a restrição de espaço físico para comportá-los. Porém, muitos desses reprovados apresentam boa qualidade, ficando com notas da pré-avaliação muito próximas dos que foram aprovados. Além disso, alguns projetos aprovados não vêem para a feira por falta de recursos financeiros para arcar com os custos do deslocamento físico.

Mais um possibilidade seria tornar a feira virtual em um “mundo virtual”, no qual a tenda seja 3D e os visitantes possam andar pelos estandes da feira.

Como esse projeto está dividido em duas etapas, na segunda serão implementadas as histórias levantadas que ainda não foram. Adicionalmente, serão feitos estudos sobre a usabilidade da rede social desenvolvida e levantadas melhorias a serem implementadas em versões posteriores para que esta se adeque melhor ao uso das pessoas. Também será estudada a questão da acessibilidade do sistema, não só para a inclusão de usuários com deficiência visual, mas também para usuários usando diferentes dispositivos para o acesso a rede social. Tendo que em vista a baixa penetração do uso de internet em dispositivos móveis, como pode ser constatado no resultados das pesquisas realizadas, projeta-se um crescimento desse uso para os próximos anos.


