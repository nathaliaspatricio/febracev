%consideracoes_finais.tex

\chapter{Considerações Finais}

A Febrace possui uma grande importância no incentivo à investigação científica e desenvolvimento tecnológico por jovens e o projeto aqui tratado extende ao mundo virtual essa atuação. Por isso, em geral, os impactos sociais decorrentes da utilização da rede social são positivos, promovendo um crescimento pessoal de seus participantes.

Na questão técnica, a metodologia de desenvolvimento se mostrou bastante aderente à necessidade do projeto e, com base em experiências anteriores em desenvolvimento de \textit{software} dos autores, apresentou vantagens em relação às metodologias mais tradicionais como o processo unificado. A escolha da metodologia foi decisiva para o sucesso do projeto, tendo em vista o curto tempo para seu desenvolvimento.

A aplicação da técnica de \textit{usability design patterns} foi adequada para a resolução de problemas na interface da Febrace\textsuperscript{V}. Como esta técnica possui uma descrição objetiva, sua adoção exigiu menor experiência para a aplicação de conceitos de usabilidade no projeto em comparação à uma avaliação com usuários.

O projeto Febrace\textsuperscript{V} têm um grande potencial de crescimento e abrangência, não só porque há cada ano estão envolvidas direta e indiretamente cerca de 13 mil pessoas com a Febrace, mas também porque parte da premissa que “tecnologias de sucesso são aquelas que se harmonizam com as necessidades e sustentam os relacionamentos e as atividades que enriquecem as experiências dos usuários” \cite{shneiderman06}.

\section{Trabalhos futuros}

São vislumbradas alguns trabalhos futuros que podem ser desenvolvidos em cima desse projeto. 

Uma dessas possíveis continuações constatada através dos dados da pesquisa de perfil de usuário é a da existência de funcionalidades que dêem suporte para que um projeto possa ser feito a distância, agregando participantes de diferentes localidades. Sendo assim, faz-se necessária uma maior integração com o sistema de submissão de projetos da Febrace, visto que projetos desenvolvidos com apoio da rede social podem ser submetidos posteriormente para a participação na feira física. 

Mais uma possibilidade seria tornar a feira virtual em um mundo virtual, no qual a tenda seja 3D e os visitantes possam criar avatares e andar pelos estandes da feira.

Também pode-se focar futuramente no estudo da questão da acessibilidade do sistema, tanto para a inclusão de usuários com deficiência visual e de usuários usando diferentes dispositivos para o acesso a rede social. Tendo que em vista a baixa penetração do uso de internet em dispositivos móveis, como pode ser constatado no resultado da pesquisas realizada, projeta-se um crescimento para os próximos anos.

Ainda há a possibilidade de usar a Febrace\textsuperscript{V} para expor concomitantemente na internet os projetos em exposição na feira física e usar o \textit{site} para a votação popular de melhor projeto.

