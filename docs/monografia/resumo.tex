%resumo.tex

\begin{resumo}

A FEBRACE (Feira Brasileira de Ciências e Engenharia), realizada todos os anos na Escola Politécnica da USP e organizada pelo Nate-LSI (Núcleo de Aprendizagem, Trabalho e Entretenimento do Laboratório de Sistemas Integráveis), é um projeto de ação contínua com o objetivo de estimular a criatividade, a reflexão, o aprofundamento e o raciocínio crítico nas atividades desenvolvidas por estudantes dos Ensinos Fundamental, Médio e Técnico, por meio da indução em realizar projetos investigativos em Ciências, Engenharia e suas aplicações.

Com o intuito de aumentar o alcance da Feira, levando-a por mais tempo a mais pessoas, e estimulando a criação de redes entre elas, o presente projeto propôs a criação de uma aplicação Web que possibilite a exposição dos projetos expostos na feira pela internet e que ofereça ferramentas que viabilizem maior interação entre os diversos envolvidos na FEBRACE (alunos participantes, professores orientadores, organizadores da Feira, avaliadores e público interessado).

Foi desenvolvida e disponibilizada uma aplicação de código aberto que oferece funcionalidades para a exposição virtual de projetos de Ciência e Engenharia. Essa ferramenta agrega também uma rede social que permite a interação entre os diversos participantes da feira e também interessados nas diversas ciências.

Além disso, foram estudados e utilizados metódos ágeis de desenvolvimento de \textit{software} e conceitos de usabilidade na realização do projeto.

\vspace{1.5ex}

{\bf Palavras-chave}: Educação, Engenharia, Feiras de ciências, Redes Sociais

\end{resumo}

\begin{abstract}

The FEBRACE (Brazilian Science and Engineering Fair), held at School of Engineering of University of São Paulo every year and organized by Nate-LSI (Learning, Work and Entertainment Center at Integrated Systems Laboratory), is a continuous action project with the objective to stimulate creativity, reflection and critical thinking in the activities developed by students from Elementary, High and Technical Schools, through induction to conduct research projects in Science, Engineering and their applications.

In order to increase the fair scope, taking it for longer time to more people, and encouraging networking between them, this project proposes a Web application creation that allows the projects exposure exhibited at the fair using the Internet and providing tools that allow greater interaction between the involved agents in FEBRACE (students, mentors, fair organizers, assessors and interested public).

It was developed and released an open source application that provides functionality for the virtual exhibit of Science and Engineering projects. This tool also adds a social network that allows the interaction between the various fair participants and also interested in the various sciences.

Furthermore, we studied and used agile software development methods and usability concepts in the project development.

\vspace{1.5ex}

{\bf Keywords}: Education, Engineering, Science Fair, Social Network

\end{abstract}
