%introducao.tex

\chapter{Introdução}

O início do século 21 vêm trazendo novos desafios para a educação. Com o advento da informática, o papel do educador deve se alterar de apenas reprodutor do conhecimento para uma postura mais provocadora, que faça com que os aprendizes sejam agentes ativos no processo de aprendizagem. 

Uma das teorias de pedagogia que partem dessa premissa é a da “aprendizagem por projetos”. Como é definido em \citeonline{fagundes99}:

\begin{quotation}
    Quando falamos em “aprendizagem por projetos” estamos necessariamente nos referindo à formulação de questões pelo autor do projeto, pelo sujeito que vai construir conhecimento. Partimos do princípio de que o aluno nunca é uma tábula rasa, isto é, partimos do princípio de que ele já pensava antes.

    E é a partir de seu conhecimento prévio, que o aprendiz vai se movimentar, interagir com o desconhecido, ou com novas situações, para se apropriar do conhecimento específico - seja nas ciências, nas artes, na cultura tradicional ou na cultura em transformação.
\end{quotation}

Dentro dessa abordagem de “aprendizagem por projetos”, uma importante e consolidada forma de incentivo é a feira de ciências. Feiras de ciências são populares no Brasil e há várias delas. Uma das mais importantes no Brasil é a FEBRACE (Feira Brasileira de Ciências e Engenharia).

A FEBRACE (Feira Brasileira de Ciências e Engenharia) é um projeto de ação contínua com o objetivo de estimular a criatividade, a reflexão, o aprofundamento e o raciocínio crítico nas atividades desenvolvidas por estudantes dos Ensinos Fundamental, Médio e Técnico, por meio da indução em realizar projetos investigativos em Ciências, Engenharia e suas aplicações \cite{lopes07}. Além disso, há uma aproximação entre as escolas públicas e privadas das Universidades, criando oportunidades de interação espontânea entre os estudantes e professores das escolas com a comunidade universitária (estudantes, professores, funcionários), para uma melhor compreensão dos papéis das Universidades em Ensino, Pesquisa, Cultura e Extensão. 

Ela é realizada todos os anos na Escola Politécnica da USP em uma tenda de eventos (figura~\ref{febrace}) e organizada pelo NATE-LSI (Núcleo de Aprendizagem, Trabalho e Entretenimento do Laboratório de Sistemas Integráveis). No ano de 2009, a feira chegou a sua 7ª edição.

    \begin{figure}
        \begin{center}
    \includegraphics[width=1.0\linewidth]{arquivos/febrace.jpg}
        \end{center}
        \caption{Tenda da FEBRACE}
        \label{febrace}
    \end{figure}

A abrangência da FEBRACE pode ser vista através da tabela~\ref{abrangencia}. Segundo estatística da organização da FEBRACE, no ano de 2009, houve a participação de 600 estudantes finalistas, 286 professores orientadores e 241 avaliadores, além de cerca de 30 voluntários no apoio à organização. A feira foi visitada por cerca de 12 mil pessoas nos três dias de exposição. 

\begin{table}[h]
    \begin{center}
        \includegraphics[width=0.8\linewidth]{arquivos/abrangencia.png}
    \end{center}
    \caption{FEBRACE em números}
    \label{abrangencia}
\end{table}

A feira conta com a participação de diversas pessoas que possuem diferentes papéis descritos abaixo:

\begin{description}
    \item[Finalistas] 
        são os estudantes que estão expondo seus projetos na feira;
    \item[Orientadores] 
        são pessoas com idade acima de 21 anos que ajudaram da orientação dos projetos expostos, podendo ser ou não professores;
    \item[Co-orientadores] 
        são pessoas com idade acima de 18 anos que ajudaram da orientação dos projetos expostos, podendo ser ou não professores e
    \item[Acompanhantes] 
        são outras pessoas que vem juntos com os alunos para acompanhá-los durante a feira (por exemplo, pais ou diretores da escola).
\end{description}

Na parte do funcionamento da feira em si, a FEBRACE não conta só com os pessoas ligadas ao Laboratório de Sistemas Integráveis da USP para sua organização. Há também uma equipe de apoio, formada geralmente por estudantes de graduação que disponibilizam de seu tempo para participar como voluntários, e uma equipe de avaliadores, que são pessoas que avaliam os projetos expostos na tenda e que devem possuir no mínimo mestrado em uma das áreas de interesse da feira.

A FEBRACE possui sete categorias, que são uma adaptação da tabela de áreas e sub-áreas do conhecimento adotada pela FAPESP (Fundação de Amparo à Pesquisa do Estado de São Paulo):

\begin{description}
    \item[Ciências Agrárias:] 
        Agronomia, Recursos Florestais e Engenharia Florestal, Engenharia Agrícola, Zootecnia, Medicina Veterinária, Recursos Pesqueiros e Engenharia de Pesca, Ciência e Tecnologia de Alimentos
    \item[Ciências Biológicas:] 
        Biologia Geral, Bioquímica, Genética, Biofísica, Botânica, Farmacologia, Zoologia, Imunologia, Ecologia, Microbiologia, Morfologia, Parasitologia, Fisiologia 	 
    \item[Ciências Exatas e da Terra:] 
        Matemática, Física, Probabilidade e Estatística, Química, Ciência da Computação, Geociências, Astronomia, Oceanografia 
    \item[Ciências Humanas:] 
        Filosofia, Geografia, Sociologia, Psicologia, Antropologia, Educação, Arqueologia, Ciência Política, História, Teologia 
    \item[Ciências da Saúde:] 
        Medicina, Odontologia, Fonoaudiologia, Farmácia, Enfermagem, Fisioterapia e Terapia Ocupacional, Nutrição, Saúde Coletiva, Educação Física
    \item[Ciências Sociais Aplicadas:] 
        Direito, Museologia, Administração, Comunicação, Economia, Serviço Social, Arquitetura e Urbanismo, Economia Doméstica, Planejamento Urbano e Regional, Desenho Industrial, Demografia, Turismo, Ciência da Informação  	 
    \item[Engenharias:] 
        Eletrônica, Sanitária, Eletrotécnica, de Produção, Mecânica, Nuclear, Química, de Transportes, Civil, Naval e Oceânica, de Minas, Aeroespacial, de Materiais e Metalúrgica, Biomédica 
\end{description}

Além de sua importância nacional, a FEBRACE é uma feira afiliada à Intel ISEF - Feira Internacional de Ciências e Engenharia - realizada anualmente em maio em diferentes cidades dos Estados Unidos da América. A Intel ISEF é a maior feira para estudantes que ainda não chegaram ao nível universitário e participam dessa feira projetos de 50 nações diferentes de todo o mundo.

Como se pode notar, a FEBRACE tem um importante papel social no estímulo a reflexão e aprendizagem de forma criativa em estudantes da educação básica através do desenvolvimento de projetos de ciências e engenharia, inclusive podendo influenciar no interesse de pessoas por uma formação técnica e/ou superior em alguma ciência. Além disso, pode influenciar no espírito de inovação dos envolvidos, podendo estimular avanços significativos em termos tecnológicos e científicos para o país.

Porém, apesar da FEBRACE ser uma forma muito interessante de incentivo, ela possui uma abrangência ainda limitada para o tamanho do território brasileiro. Infelizmente nem todos os projetos que são submetidos para a feira podem ser apresentados na tenda, já que não há espaço suficiente para comportar todos eles. Muitos desses que acabam ficando de fora na pré-avaliação eram excelentes projetos. Além disso, a FEBRACE não possui verba para financiar a vinda dos finalistas, sendo que estes devem bancar sua própria viagem e estadia em São Paulo. Não são raras as vezes em que um projeto aprovado, não conseguindo um patrocinador, acabe não participando da feira.

Por outro lado, \textit{softwares} sociais vêm crescendo muito nos últimos anos no mundo todo. Um \textit{software} social é aquele que suporta interação em grupo. Segundo \citeonline{futurelab06}, o desenvolvimento do \textit{software} social e as mudanças de objetivo na educação parecem estar indo para a mesma direção.

Tendo isso em vista, o projeto de formatura proposto visa aliar as idéias de \textit{software} social e de feira de ciências, criando assim uma feira virtual de ciências. Com o intuito de aumentar o alcance da Feira, levando-a por mais tempo a mais pessoas, e estimulando a criação de redes entre elas, é proposta a criação de uma aplicação Web que possibilite a exposição dos projetos na internet e que ofereça ferramentas que viabilizem maior interação entre os diversos envolvidos na Febrace. Além disso, será mais fácil o envolvimento de pessoas em projetos e a troca de informações entre pessoas interessadas em feiras de ciências, catalizando o papel social da FEBRACE.
