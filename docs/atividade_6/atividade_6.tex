%atividade_6.tex - Relatório de acompanhamento para a disciplina PCS20450

\documentclass[a4paper,12pt,font=plain,header=plain]{abnt}

\usepackage[brazil]{babel}
\usepackage[utf8]{inputenc}

\usepackage[num]{abntcite}
\usepackage{tabularx}

\autor{Leandro Coletto Biazon\protect\\Nathalia Sautchuk Patrício}
\titulo{Febrace\textsuperscript{V}:\\Feira Brasileira Virtual de Ciências e
Engenharia}
\orientador[Orientadoras:\\]{Profª. Drª. Roseli de Deus Lopes\protect\\Profª. Drª. Selma
Shin Shimizu Melnikoff}
\instituicao{Escola Politécnica da Universidade de São Paulo\par Departamento de Sistemas Digitais}
\local{São Paulo}
\data{Novembro de 2009}

\renewcommand{\ABNTchapterfont}{\bfseries\sffamily\fontseries{sbc}\selectfont}
\renewcommand{\ABNTsectionfont}{\bfseries\sffamily}

\begin{document}
%  \setcounter{secnumdepth}{-1}
  \capa
  %\folhaderosto

	\renewenvironment{center}{}{}
	\section*{PCS 2050 - PROJETO DE FORMATURA II}
	
	\begin{tabular}[|l|]{ |r|l| }
	\hline
				Equipe 07 & Nathalia S. Patrício, nathalia.sautchuk@gmail.com, (11)9678-1667 \\
			\hline
				& Leandro Coletto Biazon, leandrobiazon@gmail.com, (11)8961-4159 \\
			\hline
					Orientadora & Profª. Drª. Selma S. S. Melnikoff, selma.melnikoff@poli.usp.br \\
			\hline
				Co-Orientadora & Profª. Drª. Roseli de Deus Lopes, roseli@lsi.usp.br \\
			\hline
			\end{tabular} \\
	
			RA56 – Relatório de Acompanhamento (versão: 11/11/2009) \\
	
			\begin{tabular}{ |r|r|r| }
	
			\hline
				\multicolumn{3}{|c|}{Campos a serem preenchidos pelo orientador, secretaria e comitê gestor} \\
			\hline
				Orientador & Data de Entrega &  \\
			\hline
				& De acordo &  \\
			\hline
				& &  \\
			\hline
				Secretaria & Data e hora de entrega &  \\
			\hline
				&  &  \\
			\hline
				Comitê Gestor &  &  \\
			\hline
			&  &  \\
			\hline
				\multicolumn{3}{|l|}{Comentários} \\
				\multicolumn{3}{|l|}{} \\
				\multicolumn{3}{|l|}{} \\
				\multicolumn{3}{|l|}{} \\
				\multicolumn{3}{|l|}{} \\
				\multicolumn{3}{|l|}{} \\
				\multicolumn{3}{|l|}{} \\
				\multicolumn{3}{|l|}{} \\
				\multicolumn{3}{|l|}{} \\
				\multicolumn{3}{|l|}{} \\
				\multicolumn{3}{|l|}{} \\
				\multicolumn{3}{|l|}{} \\
				\multicolumn{3}{|l|}{} \\
				\multicolumn{3}{|l|}{} \\
				\multicolumn{3}{|l|}{} \\
				\multicolumn{3}{|l|}{} \\
			\hline
			\end{tabular}

	\chapter{Introdução}
	
	A Febrace (Feira Brasileira de Ciências e Engenharia), realizada todos os anos na Escola Politécnica da USP e organizada pelo Nate-LSI (Núcleo de Aprendizagem, Trabalho e Entretenimento do Laboratório de Sistemas Integráveis), é um projeto de ação contínua com o objetivo de estimular a criatividade, a reflexão, o aprofundamento e o raciocínio crítico nas atividades desenvolvidas por estudantes dos Ensinos Fundamental, Médio e Técnico, por meio da indução em realizar projetos investigativos em Ciências, Engenharia e suas aplicações.
	
	Com o intuito de aumentar o alcance da Feira, levando-a por mais tempo a mais pessoas, e estimulando a criação de redes entre elas, o presente projeto propõe a criação de uma aplicação Web que possibilite o desenvolvimento e exposição dos projetos na Internet e que ofereça ferramentas que viabilizem maior interação entre os diversos envolvidos na Febrace (alunos participantes, professores orientadores, organizadores da Feira, avaliadores e público interessado).
	
	Assim, propõe-se:
	
	\begin{itemize}
	\item{
		Desenvolver e disponibilizar uma aplicação de código aberto que ofereça ferramentas para a exposição virtual de projetos de Ciência e Engenharia;
	}
	\item{
		Agregar à exposição virtual uma rede social que permita a interação entre os participantes da feira e que estes possam se ajudar com seus projetos e dirimir dúvidas de visitantes interessados em participar de suas futuras edições e
	}
	\item{
		Estudar e utilizar conceitos de usabilidade, acessibilidade e práticas de desenvolvimento web 2.0, aplicando metódos ágeis de desenvolvimento de \textit{software}.
	}
	\end{itemize}

\chapter{Desenvolvimento}

  Essa seção descreve o desenvolvimento do projeto ocorrido entre as reuniões de acompanhamento, bem como as histórias implementadas. Ao longo do desenvolvimento, houve algumas mudanças de escopo do projeto e estas serão explicadas mais adiante.

 \section{Implementação da oitava iteração}
    A oitava iteração, com duração de três semanas, foi realizada entre 25 de setembro a 15 de outubro, e teve como resultado a implementação de uma história. A descrição de cada uma delas é feita abaixo. 

    \subsubsection{História 11 - Edição dos Conteúdos de um projeto}

      Um usuário participante de um projeto pode querer inserir mais informações sobre seu projeto. Para isso, foi criado um módulo de galeria de fotos relacionada a cada projeto, sendo que o participante pode inserir e deletar fotos desse álbum. 
      
      Há também uma interface para a visualização das fotos de um projeto e qualquer usuário tem acesso a ela.

      Um outro conteúdo adicional que pode ser inserido em um projeto é o relatório do projeto. São aceitos arquivos de texto em diferentes formatos (pdf, odt, doc).

    \subsubsection{História 21 - Estatísticas do uso do sistema}

        Foi criada uma página simples que mostra algumas estatísticas do sistema. Essa página é de acesso público, sendo que não precisa estar logado para ver seu conteúdo. São mostradas quatro estatísticas:
	\begin{itemize}
	  \item{
		Projetos mais seguidos
	  }
	  \item{
		Usuários mais seguidos
	  }
	  \item{ 
	    Projetos mais comentados 
	  }
	  \item{ 
	    Posts mais comentados
	  }
	\end{itemize}

        A idéia é a de mostrar a relevância dos conteúdos existentes na rede social.

 \section{Redefinição de escopo}

    Juntamente com a orientadora do projeto e com o cliente, foi avaliado o andamento do trabalho e foi redefinido o escopo do projeto com base em prioridades. Assim, quatro cartões inicialmente levantados foram suprimidos:

      \begin{itemize}
	\item Convite automático a ex-participantes para participação na rede social: esse cartão está relacionado com a fase de implantação do projeto e não com seu desenvolvimento 
	\item Postar em fórum: esse cartão foi considerado de baixa prioridade e com baixo valor agregado, já que a rede social começará com poucos usuários e para se ter um fórum de relevância é necessário ter uma grande massa de dados. 
	\item Moderação de conteúdos: esse cartão foi considerado de baixa prioridade e com baixo valor agregado, já que a rede social começará com poucos usuários. 
	\item Estatísticas do servidor que roda o sistema: esse cartão está relacionado com a fase de implantação do projeto e não com seu desenvolvimento 
      \end{itemize}

  	\chapter{Cronograma}

	\begin{tabularx}{0.9\linewidth}[|l|]{ |r|X|l|X| }
	\hline
		\multicolumn{2}{|c|}{\textbf{Atividades planejadas}} \\
	\hline
		16/10-30/11 & Estudos de usabilidade \\
	\hline
		01/11-15/11 & Confecção do Pôster, \textit{Press Releases} e Página de Internet \\
	\hline
		11/11 & Entrega do 6º documento de acompanhamento \\
	\hline
		16/11 & Entrega do Pôster, \textit{Press Releases} e Página de Internet \\
	\hline
		01/12-06/12 & Finalização da monografia do Projeto de Formatura \\
	\hline
		07/12 & Entrega da monografia do Projeto de Formatura \\
	\hline
		14/12 & Demonstração prática do projeto de Formatura no PCS \\
	\hline
		15/12 & Banca do Projeto de Formatura no PCS \\
	\hline
	\end{tabularx} \\

\end{document}
