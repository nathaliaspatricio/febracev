%atividade_3.tex - Relatório de acompanhamento do projeto para a disciplina PCS2040

\documentclass[a4paper,12pt,font=plain,header=plain]{abnt}

\usepackage[brazil]{babel}
\usepackage[utf8]{inputenc}

\autor{Leandro Coletto Biazon\protect\\Nathalia Sautchuk Patrício}
\titulo{Febrace\textsuperscript{V}:\\Feira Brasileira Virtual de Ciências e
Engenharia}
\orientador[Orientadores:\\]{Profª. Drª. Roseli de Deus Lopes\protect\\Profª. Drª. Selma
Shin Shimizu Melnikoff}
\instituicao{Escola Politécnica da Universidade de São Paulo\par Departamento de Sistemas Digitais}
\local{São Paulo}
\data{Março de 2009}

\renewcommand{\ABNTchapterfont}{\bfseries\sffamily\fontseries{sbc}\selectfont}
\renewcommand{\ABNTsectionfont}{\bfseries\sffamily}

\begin{document}
  \setcounter{secnumdepth}{-1}
  \capa
  %\folhaderosto

  \renewenvironment{center}{}{}
	\section{PCS 2040 - PROJETO DE FORMATURA I}
 
		\begin{tabular}[|l|]{ |r|l| }
		\hline
			Equipe 07 & Nathalia S. Patrício, nathalia.sautchuk@gmail.com, (11)9678-1667 \\
		\hline
			& Leandro Coletto Biazon, leandrobiazon@gmail.com, (11)8961-4159 \\
		\hline
			Orientador & Prof. Dr. Selma S. S. Melnikoff, selma.melnikoff@poli.usp.br \\
		\hline
			Co-Orientador & Prof. Dr. Roseli de Deus Lopes, roseli@lsi.usp.br \\
		\hline
		\end{tabular} \\
		
		RA1 – Relatório Resumido (versão: 20/03/2009) \\
		
		\begin{tabular}{ |r|r|r| }
	
		\hline
			\multicolumn{3}{|c|}{Campos a serem preenchidos pelo orientador, secretaria e comitê gestor} \\
		\hline
			Orientador & Data de Entrega &  \\
		\hline
			& De acordo &  \\
		\hline
			& &  \\
		\hline
			Secretaria & Data e hora de entrega &  \\
		\hline
			&  &  \\
		\hline
			Comitê Gestor &  &  \\
		\hline
			&  &  \\
		\hline
			\multicolumn{3}{|l|}{Comentários} \\
			\multicolumn{3}{|l|}{} \\
			\multicolumn{3}{|l|}{} \\
			\multicolumn{3}{|l|}{} \\
			\multicolumn{3}{|l|}{} \\
			\multicolumn{3}{|l|}{} \\
			\multicolumn{3}{|l|}{} \\
			\multicolumn{3}{|l|}{} \\
			\multicolumn{3}{|l|}{} \\
			\multicolumn{3}{|l|}{} \\
			\multicolumn{3}{|l|}{} \\
			\multicolumn{3}{|l|}{} \\
			\multicolumn{3}{|l|}{} \\
			\multicolumn{3}{|l|}{} \\
			\multicolumn{3}{|l|}{} \\
			\multicolumn{3}{|l|}{} \\
		\hline
		\end{tabular}

  	\section{Atividades Realizadas no Período}
		
		Como dito no documento de acompanhamento anterior, para a escrita dessa documento foi realizada a 2ª retrospectiva.

		Nesse período entre a 1ª e a 2ª retrospectiva, foram feitas as seguintes atividades:
		\begin{itemize}
		 \item questionário de perfil de uso: devido a algumas sugestões recebidas da equipe da FEBRACE e de outras pessoas com experiência em levantamento de dados com usuários foram feitos melhoramentos no questionário;
		 \item aplicação do questionário: a FEBRACE ocorreu nos dias 17, 18 e 19 de março e durante esse período foram aplicados questionários para alunos e professores participantes, além de visitantes da feira. Ainda não houve tempo para a compilação dos dados dos questionários;
		 \item automação do processamento de dados coletados com os questionários: o método escolhido para a aplicação do questionário foi em papel. Porém, com o objetivo de facilitar a compilação desses dados será usada a ferramenta Lime Survey. Assim, os dados de cada questionário podem ser inseridos nessa ferramenta de forma mais rápida e prática, e, pelo fato de ser online, o trabalho poder ser mais facilmente entre os membros do grupo. Além disso, a ferramenta já faz gráficos e cálculos de desvio padrão;
		 \item decisão das tecnologias a serem utilizadas no projeto: o projeto será desenvolvido em plataforma Linux, usando o servidor web Apache. A linguagem escolhida foi Python, com o uso do framework Django para a construção de aplicações web. Serão usadas ferramentas para teste automatizado de código como o PyUnit, o Twil e o Selenium. Uma possibilidade levantada pelo grupo é a do uso de componentes reusáveis do Django Plugables. O banco de dados a ser utilizado ainda está em aberto, sendo que para se decidir serão testados os desempenhos dos bancos de dados MySQL e Postgre.
		\end{itemize}
	
  	\section{Planejamento e Cronograma de Atividades}

		Até o momento o cronograma está sendo seguido e o grupo não se encontra atrasado em relação ao mesmo. Ainda sim houve um replanejamento, pois algumas atividades ficaram mais claras nesse último mês e de acordo com sugestões da orientação.					
		
		\begin{tabular}[|l|]{ |r|l| }
		\hline
			\multicolumn{2}{|c|}{\textbf{Cronograma}} \\
		\hline
			14/03 & 2ª retrospectiva \\
		\hline
			15/03-21/03 & aplicação do questionário de levantamento de perfil \\
		\hline
			20/03 & entrega do 2º documento de acompanhamento \\
		\hline
			23/03 & 3ª apresentação \\
		\hline
			22/03-28/03 & compilação e análise dos dados do questionário \\
			 & levantamento de histórias através Planning Game com o cliente \\ 
			 & e definição da 1ª iteração \\
		\hline
			29/03-04/04 & implementação \\
		\hline
			05/03-11/04 & implementação \\
		\hline
			11/04 & 3ª retrospectiva \\
		\hline
			12/03-18/04 & implementação \\
		\hline
			17/04 & entrega do 3º documento de acompanhamento \\
		\hline
			23/04 & 4ª apresentação \\
		\hline
			19/04-20/06 & implementação \\
		\hline
			21/06-27/06 & revisão final \\
		\hline
			30/06 & entrega da monografia 1 e projeto 1 no PSI \\
		\hline
			01/07-31/08 & testes de usabilidade e avaliação \\
		\hline
			01/09-27/11 & implementação com base nos resultados dos testes \\
		\hline
			30/11 & entrega da monografia 2 e projeto 2 no PCS \\
		\hline
		\end{tabular} \\

\end{document}