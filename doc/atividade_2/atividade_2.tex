%atividade_2.tex - Relatório de acompanhamento do projeto para a disciplina PCS2040

\documentclass[a4paper,12pt,font=plain,header=plain]{abnt}

\usepackage[brazil]{babel}
\usepackage[utf8]{inputenc}

\autor{Leandro Coletto Biazon\protect\\Nathalia Sautchuk Patrício}
\titulo{Febrace\textsuperscript{V}:\\Feira Brasileira Virtual de Ciências e
Engenharia}
\orientador[Orientadores:\\]{Profª. Drª. Roseli de Deus Lopes\protect\\Profª. Drª. Selma
Shin Shimizu Melnikoff}
\instituicao{Escola Politécnica da Universidade de São Paulo\par Departamento de Sistemas Digitais}
\local{São Paulo}
\data{Fevereiro de 2009}

\renewcommand{\ABNTchapterfont}{\bfseries\sffamily\fontseries{sbc}\selectfont}
\renewcommand{\ABNTsectionfont}{\bfseries\sffamily}

\begin{document}
  \setcounter{secnumdepth}{-1}
  \capa
  %\folhaderosto

  \renewenvironment{center}{}{}

  \section{Objetivos}
    O projeto tem como objetivo desenvolver uma rede social focada em desenvolvimento e exposição de projetos de ciências e engenharia. Dentre os conceitos de engenharia de software a serem aplicados no projeto, destacam-se a temática de usabilidade e acessibilidade, metódos ágeis de desenvolvimento de software e práticas de desenvolvimento web.

  \section{Resumo}

    A Febrace (Feira Brasileira de Ciências e Engenharia), realizada todos os
    anos na Escola Politécnica da USP e organizada pelo Nate-LSI (Núcleo de
    Aprendizagem, Trabalho e Entretenimento do Laboratório de Sistemas
    Integráveis), é um projeto de ação contínua com o objetivo de estimular a
    criatividade, a reflexão, o aprofundamento e o raciocínio crítico nas
    atividades desenvolvidas por estudantes dos Ensinos Fundamental, Médio e
    Técnico, por meio da indução em realizar projetos investigativos em
    Ciências, Engenharia e suas aplicações.

    Com o intuito de aumentar o alcance da Feira, levando-a por mais tempo a
    mais pessoas, e estimulando a criação de redes entre elas, o presente
    projeto propõe a criação de uma aplicação Web que possibilite o
    desenvolvimento e exposição dos projetos na Internet e que ofereça
    ferramentas que viabilizem maior interação entre os diversos envolvidos na
    Febrace (alunos participantes, professores orientadores, organizadores da
    Feira, avaliadores e público interessado).

    Assim, propõe-se:

    \begin{itemize}
      \item{
        Desenvolver e disponibilizar uma aplicação de código aberto que ofereça ferramentas para a exposição virtual de projetos de Ciência e Engenharia;
      }
      \item{
        Agregar à exposição virtual uma rede social que permita a interação entre os participantes da feira e que estes possam se ajudar com seus projetos e dirimir dúvidas de visitantes interessados em participar de suas futuras edições e
      }
      \item{
        Estudar e utilizar conceitos de usabilidade, acessibilidade e práticas de desenvolvimento web 2.0, aplicando metódos ágeis de desenvolvimento de software.
      }
    \end{itemize}

  \section{Atividades Realizadas no Período}
	
	Tendo os objetivos postos acima do projeto de formatura em questão foram levantadas algumas atividades iniciais essenciais a serem realizadas durante o período de cerca de um mês ocorrido entre a proposta inicial e esta atividade de acompanhamento.
	
	O primeiro passo tomado pelo grupo foi o estudo dos conceitos de engenharia de software envolvidos na proposta como metodologias agéis, usabilidade e acessibilidade e desenvolvimento web. Esse estudo teve como principal objetivo ter maior contato com os esses conceitos, levantar uma bibliografia inicial para o projeto de formatura e definir os próximos passos.

	Como um dos desdobramentos do primeiro passo percebeu-se a necessidade da adaptação da proposta das metodologias agéis de desenvolvimento de software ao contexto do projeto de formatura, visto que alguns documentos são necessários. Foi feito um primeiro esboço de uma proposta de adaptação a metodologia. Como este é um assunto novo em termos científicos (começou a ser desenvolvido a pouco mais de 10 anos) definiu-se que um dos focos principais do projeto será fazer um relato do processo de desenvolvimento da aplicação proposta, descrevendo com detalhes as adaptações necessárias no modelo para que ele se adeque a um contexto acadêmico. Uma decisão que foi tomada é que a monografia será escrita concomitatemente com o andamento do projeto devido a este objetivo.
	
	Uma das adaptações que já está começando a ser feita na metodologia diz respeito à retrospectiva. A restropectiva é uma reunião periódica na qual é avaliado o período anterior do projeto (entre aquela retrospectiva e a anterior). Nessa reunião participa toda a equipe de desenvolvimento e a idéia é levantar coisas que deram certo naquele período, coisas que precisam ser melhoradas e idéias (desde do projeto em si até coisas referentes ao ambiente de trabalho) que possam ter surgidos durante esse período. Não há um tempo pré-determinado entre retrospectivas, mas para equipes que trabalham em período integral juntas é aconselhável que sejam quinzenais, enquanto não é ideal que demorem mais que um mês para ocorrer. Como uma adaptação ao processo, foi decidido que no caso do projeto em questão essas restrospectivas ocorrerão mensalmente e com base nelas serão escritos os documentos de acompanhamento a serem entregues na disciplina.

	O outro desdobramento do primeiro passo veio através do estudo dos conceitos de usabilidade. Para se fazer um estudo válido de usabilidade verificou-se como necessário um levantamento prévio com possíveis usuários do sistema para verificar o seu perfil de uso e suas reais necessidades em relação a uma nova ferramenta computacional. Como a FEBRACE (Feira Brasileira de Ciências e Engenharia) irá ocorrer em breve (dias 17 a 19 de março) viu-se a possibilidade de fazer esse levantamento com os participantes da feira que são potenciais usuários da rede social. O método escolhido para fazer esse levantamento foi através de questionário e este já foi elaborado, estando no anexo 1.

	Outra atividade realizada foi um planejamento preliminar de todo o projeto de formatura, porém este foi feito em linhas gerais. Uma das propostas das metodologias agéis é o constante replanejamento para que se consiga prever com maior antecedência se o projeto está sofrendo problemas e ter tempo para tomar decisões para correção. Outra proposta também dessa metodologia é que um planejamento só pode ser feito com uma boa precisão para poucas semanas adiante da que você está agora. Aliando-se essas duas propostas temos o constante replanejamento (que deve ocorrer sempre que necessário) e a cada novo replanejamento as atividades das semanas seguintes são melhores especificadas.

  \section{Planejamento e Cronograma de Atividades}

	Como já foi dito na seção anterior, este é um planejamento preliminar e portanto o cronograma irá sofrer alterações conforme ocorra o replanejamento.

	No cronograma não consta, mas há a escrita de tópicos da monografia referentes a cada etapa do desenvolvimento do projeto.

	O projeto está dividido em duas partes, pois os integrantes estão matriculados em diferentes matérias de projeto de formatura, ou seja, um está matriculado em Projeto de Formatura 2 e o outro em Projeto de Formatura 1. Assim, na primeira parte haverá a entrega de uma monografia preliminar e um projeto funcional, que serão apresentados no PSI. Mesmo assim na parte 2 ambos os integrantes irão participar do desenvolvimento.
	
	\begin{tabular}[|l|]{ |r|l| }
	\hline
		\multicolumn{2}{|c|}{\textbf{Cronograma}} \\
	\hline
		27/02 & 1ª retrospectiva e entrega do 2º documento de acompanhamento \\
	\hline
		02/03 & 2ª apresentação do projeto \\
	\hline
		01/03-07/03 & estudo e definição da tecnologia \\
	\hline
		08/03-14/03 & levantamento de histórias iniciais \\
	\hline
		15/03-21/03 & aplicação do questionário de levantamento de perfil \\
	\hline
		20/03 & 2ª retrospectiva e entrega do 2º documento de acompanhamento \\
	\hline
		23/03 & 3ª apresentação \\
	\hline
		22/03-28/03 & análise dos dados do questionário \\
	\hline
		29/03-04/04 & implementação \\
	\hline
		05/03-11/04 & implementação \\
	\hline
		12/03-18/04 & implementação \\
	\hline
		17/04 & retrospectiva e entrega do 3º documento de acompanhamento \\
	\hline
		23/04 & 4ª apresentação \\
	\hline
		19/04-20/06 & implementação \\
	\hline
		21/06-27/06 & revisão final \\
	\hline
		30/06 & entrega da monografia 1 e projeto 1 no PSI \\
	\hline
		01/07-31/08 & testes de usabilidade e avaliação \\
	\hline
		01/09-27/11 & implementação com base nos resultados dos testes \\
	\hline
		30/11 & entrega da monografia 2 e projeto 2 no PCS \\
	\hline
	\end{tabular} \\

\end{document}
