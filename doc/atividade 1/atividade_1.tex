%atividade_1.tex - Dados básicos do projeto para a disciplina PCS2040

\documentclass[a4paper,12pt,font=plain,header=plain]{abnt}

\usepackage[brazil]{babel}
\usepackage[utf8]{inputenc}
\usepackage{graphicx}
\usepackage{natbib}

\autor{Leandro Coletto Biazon\protect\\Nathalia Sautchuk Patrício}
\titulo{Febrace\textsuperscript{V}:\\Feira Brasileira Virtual de Ciências e
Engenharia}
\orientador[Orientadores:\\]{Profª. Drª. Roseli de Deus Lopes\protect\\Profª. Drª. Selma
Shin Shimizu Melnikoff}
\instituicao{Escola Politécnica da Universidade de São Paulo\par Departamento de Sistemas Digitais}
\local{São Paulo}
\data{Janeiro de 2009}

\renewcommand{\ABNTchapterfont}{\bfseries\sffamily\fontseries{sbc}\selectfont}
\renewcommand{\ABNTsectionfont}{\bfseries\sffamily}

\begin{document}
  \setcounter{secnumdepth}{-1}
  \capa
  \folhaderosto

  \renewenvironment{center}{}{}
  \section{Objetivos}
    O projeto tem como objetivos especificar e implementar uma rede social
    focada em desenvolvimento e exposição de projetos de ciências e engenharia.
    Dentre os conceitos de engenharia serão abordados a temática de usabilidade
    e acessibilidade, metódos ágeis de desenvolvimento de software e práticas de
    desenvolvimento web.

  \section{Resumo}
  %Premissas:
  %
  %Engenharia
  %
  %- Usabilidade e acessibilidade (IHC)
  %- Engenharia de software -> Metodologias ágeis (XP, Scrum), Desenvolvimento
  %Web.
  %- Software Social
  %
  %Outros
  %
  %- Produção social, Software Livre
  %- Febrace
  %- Pedagogia de projetos

    A Febrace (Feira Brasileira de Ciências e Engenharia), realizada todos os
    anos na Escola Politécnica da USP e organizada pelo Nate-LSI (Núcleo de
    Aprendizagem, Trabalho e Entretenimento do Laboratório de Sistemas
    Integráveis), é um projeto de ação contínua com o objetivo de estimular a
    criatividade, a reflexão, o aprofundamento e o raciocínio crítico nas
    atividades desenvolvidas por estudantes dos Ensinos Fundamental, Médio e
    Técnico, por meio da indução em realizar projetos investigativos em
    Ciências, Engenharia e suas aplicações.

    Com o intuito de aumentar o alcance da Feira, levando-a por mais tempo a
    mais pessoas, e estimulando a criação de redes entre elas, o presente
    projeto propõe a criação de uma aplicação Web que possibilite o
    desenvolvimento e exposição dos projetos na Internet
    \cite{andersson_software_2006}, e que ofereça
    ferramentas que viabilizem maior interação entre os diversos envolvidos na
    Febrace (alunos participantes, professores orientadores, organizadores da
    Feira, avaliadores e público interessado).

    \begin{itemize}
      \item{
        Desenvolver e disponibilizar um aplicação de código aberto que
        ofereça ferramentas para a exposição virtual de projetos de Ciência e
        Engenharia e para fins similares;
      }
      \item{
        Agregar à infraestrutura de exposição de projetos funcionalidades que
        permitam a formação de uma comunidade de aprendizagem online, na qual os
        usuários, em maioria atuais participantes da Feira, possam se ajudar com
        seus projetos e dirimir dúvidas de visitantes interessados em participar
        de suas futuras edições;
      }
    \end{itemize}

  \section{Integrantes e orientadores}
    Os integrantes são alunos dos Departamentos de Computação e Sistemas Digitais
    (PCS) e de Sistemas Eletrônicos (PSI).

    \begin{itemize}
      \item Leandro Coletto Biazon (PSI)
      \item Nathalia Sautchuk Patrício (PCS)
    \end{itemize}

    \subsection{Orientadores}
    \begin{itemize}
    \item{
      Profª. Drª. Roseli de Deus Lopes \\
      (LSI – Laboratório de Sistemas Integráveis)
    }
    \item{
      Profª. Drª. Selma Shin Shimizu Melnikoff \\
      (LTS – Laboratório de Tecnologia de Software)
    }
    \end{itemize}

  \bibliography{atividade_1}{}
  \bibliographystyle{plain}

\end{document}
