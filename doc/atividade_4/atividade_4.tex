%atividade_4.tex - Relatório de especificação do projeto para a disciplina PCS2040

\documentclass[a4paper,12pt,font=plain,header=plain]{abnt}

\usepackage[brazil]{babel}
\usepackage[utf8]{inputenc}

\autor{Leandro Coletto Biazon\protect\\Nathalia Sautchuk Patrício}
\titulo{Febrace\textsuperscript{V}:\\Feira Brasileira Virtual de Ciências e
Engenharia}
\orientador[Orientadoras:\\]{Profª. Drª. Roseli de Deus Lopes\protect\\Profª. Drª. Selma
Shin Shimizu Melnikoff}
\instituicao{Escola Politécnica da Universidade de São Paulo\par Departamento de Sistemas Digitais}
\local{São Paulo}
\data{Abril de 2009}

\renewcommand{\ABNTchapterfont}{\bfseries\sffamily\fontseries{sbc}\selectfont}
\renewcommand{\ABNTsectionfont}{\bfseries\sffamily}

\begin{document}
  \setcounter{secnumdepth}{-1}
  \capa
  %\folhaderosto

  \renewenvironment{center}{}{}
  \section{PCS 2040 - PROJETO DE FORMATURA I}

    \begin{tabular}[|l|]{ |r|l| }
    \hline
			Equipe 07 & Nathalia S. Patrício, nathalia.sautchuk@gmail.com, (11)9678-1667 \\
		\hline
			& Leandro Coletto Biazon, leandrobiazon@gmail.com, (11)8961-4159 \\
		\hline
			Orientadora & Profª. Drª. Selma S. S. Melnikoff, selma.melnikoff@poli.usp.br \\
		\hline
			Co-Orientadora & Profª. Drª. Roseli de Deus Lopes, roseli@lsi.usp.br \\
		\hline
		\end{tabular} \\

		RA1 – Relatório de Especificação de Software (versão: 12/04/2009) \\

		\begin{tabular}{ |r|r|r| }

		\hline
			\multicolumn{3}{|c|}{Campos a serem preenchidos pelo orientador, secretaria e comitê gestor} \\
		\hline
			Orientador & Data de Entrega &  \\
		\hline
			& De acordo &  \\
		\hline
			& &  \\
		\hline
			Secretaria & Data e hora de entrega &  \\
		\hline
			&  &  \\
		\hline
			Comitê Gestor &  &  \\
		\hline
			&  &  \\
		\hline
			\multicolumn{3}{|l|}{Comentários} \\
			\multicolumn{3}{|l|}{} \\
			\multicolumn{3}{|l|}{} \\
			\multicolumn{3}{|l|}{} \\
			\multicolumn{3}{|l|}{} \\
			\multicolumn{3}{|l|}{} \\
			\multicolumn{3}{|l|}{} \\
			\multicolumn{3}{|l|}{} \\
			\multicolumn{3}{|l|}{} \\
			\multicolumn{3}{|l|}{} \\
			\multicolumn{3}{|l|}{} \\
			\multicolumn{3}{|l|}{} \\
			\multicolumn{3}{|l|}{} \\
			\multicolumn{3}{|l|}{} \\
			\multicolumn{3}{|l|}{} \\
			\multicolumn{3}{|l|}{} \\
		\hline
		\end{tabular}

  \section{Objetivos}
    O projeto tem como objetivo desenvolver uma rede social focada em desenvolvimento e exposição de projetos de ciências e engenharia. Dentre os conceitos de engenharia de software a serem aplicados no projeto, destacam-se a temática de usabilidade e acessibilidade, metódos ágeis de desenvolvimento de software e práticas de desenvolvimento web.

  \section{Resumo}

    A Febrace (Feira Brasileira de Ciências e Engenharia), realizada todos os anos na Escola Politécnica da USP e organizada pelo Nate-LSI (Núcleo de Aprendizagem, Trabalho e Entretenimento do Laboratório de Sistemas Integráveis), é um projeto de ação contínua com o objetivo de estimular a criatividade, a reflexão, o aprofundamento e o raciocínio crítico nas atividades desenvolvidas por estudantes dos Ensinos Fundamental, Médio e Técnico, por meio da indução em realizar projetos investigativos em Ciências, Engenharia e suas aplicações.

    Com o intuito de aumentar o alcance da Feira, levando-a por mais tempo a mais pessoas, e estimulando a criação de redes entre elas, o presente projeto propõe a criação de uma aplicação Web que possibilite o desenvolvimento e exposição dos projetos na Internet e que ofereça ferramentas que viabilizem maior interação entre os diversos envolvidos na Febrace (alunos participantes, professores orientadores, organizadores da Feira, avaliadores e público interessado).

    Assim, propõe-se:

    \begin{itemize}
      \item{
        Desenvolver e disponibilizar uma aplicação de código aberto que ofereça ferramentas para a exposição virtual de projetos de Ciência e Engenharia;
      }
      \item{
        Agregar à exposição virtual uma rede social que permita a interação entre os participantes da feira e que estes possam se ajudar com seus projetos e dirimir dúvidas de visitantes interessados em participar de suas futuras edições e
      }
      \item{
        Estudar e utilizar conceitos de usabilidade, acessibilidade e práticas de desenvolvimento web 2.0, aplicando metódos ágeis de desenvolvimento de software.
      }
    \end{itemize}

  \section{Descrição}

  \section{Tecnologias e justificativa}
    O projeto será desenvolvido em plataforma Linux, usando o servidor web Apache. A linguagem de programação escolhida foi o Python, com o uso do framework Django para a construção de aplicações web. Serão usadas ferramentas para teste automatizado de código como o PyUnit, o Twil e o Selenium. Uma possibilidade levantada pelo grupo é a do uso de componentes reusáveis do Django Plugables. O banco de dados a ser utilizado ainda está em aberto, sendo que para se decidir serão testados os desempenhos dos bancos de dados MySQL e PostgreSQL.

    As tecnologias foram escolhidas com base na experiência e habilidades técnicas da equipe.

  \section{Extreme Programming}
	  Como um dos desdobramentos do primeiro passo percebeu-se a necessidade da adaptação da proposta das metodologias agéis de desenvolvimento de software ao contexto do projeto de formatura, visto que alguns documentos são necessários. Foi feito um primeiro esboço de uma proposta de adaptação a metodologia. Como este é um assunto novo em termos científicos (começou a ser desenvolvido a pouco mais de 10 anos) definiu-se que um dos focos principais do projeto será fazer um relato do processo de desenvolvimento da aplicação proposta, descrevendo com detalhes as adaptações necessárias no modelo para que ele se adeque a um contexto acadêmico. Uma decisão que foi tomada é que a monografia será escrita concomitatemente com o andamento do projeto devido a este objetivo.

    Uma das adaptações que já está começando a ser feita na metodologia diz respeito à retrospectiva. A restropectiva é uma reunião periódica na qual é avaliado o período anterior do projeto (entre aquela retrospectiva e a anterior). Nessa reunião participa toda a equipe de desenvolvimento e a idéia é levantar coisas que deram certo naquele período, coisas que precisam ser melhoradas e idéias (desde do projeto em si até coisas referentes ao ambiente de trabalho) que possam ter surgidos durante esse período. Não há um tempo pré-determinado entre retrospectivas, mas para equipes que trabalham em período integral juntas é aconselhável que sejam quinzenais, enquanto não é ideal que demorem mais que um mês para ocorrer. Como uma adaptação ao processo, foi decidido que no caso do projeto em questão essas restrospectivas ocorrerão mensalmente e com base nelas serão escritos os documentos de acompanhamento a serem entregues na disciplina.

    \subsection{Metodologias ágeis no contexto acadêmico}
      Uma das metas do presente projeto é testar a validade de um conjunto de práticas propostas pelas metodologias ágeis de desenvolvimento de software, e experimentar sua consistência quando aplicada no contexto acadêmico. Objetiva-se também documentar essa experiência de forma que outros alunos que também queiram trabalhar com essas metodologias em seus projetos na universidade tenham um relato no qual se basear, com possíveis heurísticas e adaptações que se fizeram necessárias no nosso caso particular.

      Tendo isso em vista, realizou-se uma pesquisa por artigos que descrevessem experiências semelhantes de aplicação de metodologias ágeis na graduação. Foram encontrados diversos relatos dessa natureza, muitos deles descrevendo a utilização dessas metodologias em projetos de conclusão de curso, muito convenientes por se situarem no mesmo contexto em que estamos inseridos.

  \section{Arquitetura do sistema}

  \section{Cartões e 1\textsuperscript{a} iteração}

  \section{Cronograma}

	\begin{tabular}[|l|]{ |r|l| }
	\hline
		\multicolumn{2}{|c|}{\textbf{Cronograma}} \\
	\hline
		27/02 & 1ª retrospectiva e entrega do 2º documento de acompanhamento \\
	\hline
		02/03 & 2ª apresentação do projeto \\
	\hline
		01/03-07/03 & estudo e definição da tecnologia \\
	\hline
		08/03-14/03 & elaboração de questionário de levantamento de perfil \\
	\hline
		14/03 & 2ª retrospectiva \\
	\hline
		15/03-21/03 & aplicação do questionário de levantamento de perfil \\
	\hline
		20/03 & entrega do 2º documento de acompanhamento \\
	\hline
		23/03 & 3ª apresentação \\
	\hline
		22/03-28/03 & compilação e análise dos dados do questionário \\
	\hline
		29/03-04/04 & levantamento de histórias através Planning Game com o cliente \\
			    & e definição da 1ª iteração \\
	\hline
		05/03-11/04 & implementação da 1ª iteração \\
	\hline
		11/04 & 3ª retrospectiva \\
	\hline
		12/03-18/04 & implementação da 1ª iteração \\
	\hline
		17/04 & entrega do 3º documento de acompanhamento \\
	\hline
		19/04-25/04 & implementação da 1ª iteração \\
	\hline
		23/04 & 4ª apresentação \\
	\hline
		26/04-16/05 & implementação da 2ª iteração \\
	\hline
		08/05 & entrega do relatório parcial de atividades no PSI \\
	\hline
		15/05 & apresentação de atividades no PSI \\
	\hline
		17/05-06/06 & implementação da 3ª iteração \\
	\hline
		07/06-25/06 & implementação da 4ª iteração \\
	\hline
		14/06-18/06 & revisão final da monografia 1 \\
	\hline
		19/06 & entrega da monografia 1 no PSI \\
	\hline
		26/06 & apresentação do projeto 1 no PSI \\
	\hline
		01/07-31/08 & testes de usabilidade e avaliação \\
	\hline
		01/09-27/11 & implementação com base nos resultados dos testes \\
	\hline
		30/11 & entrega da monografia 2 e projeto 2 no PCS \\
	\hline
	\end{tabular} \\

\end{document}
