%introducao.tex

\chapter{Introdução}

  A Febrace (Feira Brasileira de Ciências e Engenharia), realizada todos os anos na Escola Politécnica da USP e organizada pelo Nate-LSI (Núcleo de Aprendizagem, Trabalho e Entretenimento do Laboratório de Sistemas Integráveis), é um projeto de ação contínua com o objetivo de estimular a criatividade, a reflexão, o aprofundamento e o raciocínio crítico nas atividades desenvolvidas por estudantes dos Ensinos Fundamental, Médio e Técnico, por meio da indução em realizar projetos investigativos em Ciências, Engenharia e suas aplicações\cite{lopes07}.

  Com o intuito de aumentar o alcance da Feira, levando-a por mais tempo a mais pessoas, e estimulando a criação de redes entre elas, o presente projeto propõe a criação de uma aplicação Web que possibilite o desenvolvimento e exposição dos projetos na Internet e que ofereça ferramentas que viabilizem maior interação entre os diversos envolvidos na Febrace (alunos participantes, professores orientadores, organizadores da Feira, avaliadores e público interessado).

  Assim, propõe-se:

  \begin{itemize}
    \item{
      Desenvolver e disponibilizar uma aplicação de código aberto que ofereça ferramentas para a exposição virtual de projetos de Ciência e Engenharia;
    }
    \item{
		Agregar à exposição virtual uma rede social que permita a interação entre os participantes da feira e que estes possam se ajudar com seus projetos e dirimir dúvidas de visitantes interessados em participar de suas futuras edições e
    }
    \item{
		Estudar e utilizar conceitos de usabilidade, acessibilidade e práticas de desenvolvimento web 2.0, aplicando metódos ágeis de desenvolvimento de \textit{software}.
    }
  \end{itemize}

  \section{Objetivo}
    O projeto tem como objetivo desenvolver uma rede social focada em desenvolvimento e exposição de projetos de ciências e engenharia. Dentre os conceitos de engenharia de \textit{software} a serem aplicados no projeto, destacam-se a temática de usabilidade e acessibilidade, metódos ágeis de desenvolvimento de \textit{software} e práticas de desenvolvimento web.
